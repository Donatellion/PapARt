\documentclass{article}

\usepackage{times}
\usepackage{uist}

\begin{document}

% --- Copyright notice ---
\conferenceinfo{UIST'12}{October 7-10, 2012, Cambridge, MA, USA}
\CopyrightYear{2012}
\crdata{978-1-xxxx-xxxx-x}

% Uncomment the following line to hide the copyright notice
% \toappear{}
% ------------------------

\bibliographystyle{plain}

\title{Demo: Spatial augmented reality for \\
       physical drawing}

%%
%% Note on formatting authors at different institutions, as shown below:
%% Change width arg (currently 7cm) to parbox commands as needed to
%% accommodate widest lines, taking care not to overflow the 17.8cm line width.
%% Add or delete parboxes for additional authors at different institutions. 
%% If additional authors won't fit in one row, you can add a "\\"  at the
%% end of a parbox's closing "}" to have the next parbox start a new row.
%% Be sure NOT to put any blank lines between parbox commands!
%%

\author{
\parbox[t]{9cm}{\centering
	     {\em Jeremy Laviole}\\
	     Univ. Bordeaux, LaBRI, UMR 5800, F-33400 Talence, France.\\
         CNRS, LaBRI, UMR 5800, F-33400 Talence, France.\\
	     Inria, F-33400 Talence, France.\\
	     laviole@labri.fr}
	     
\parbox[t]{9cm}{\centering
	     {\em Martin Hachet}\\
	     Inria, F-33400 Talence, France.\\
	     LaBRI, UMR 5800, F-33400 Talence, France.\\
	     martin.hachet@inria.fr}
}


\maketitle

\abstract
Spatial augmented reality (SAR) allows the projection of virtual environments into the real world. In this demo, we propose to demonstrate our SAR tools dedicated to drawing. 
From the most simple tools: the projection on virtual guidelines enabling to trace lines and curves just by following the guides to more advanced techniques enabling stereoscopic drawing. This demo presents how we can use computer graphics tools to ease the drawing, and how it will enable new forms of physical drawing. 

\classification{H5.2 [Information interfaces and presentation]:
User Interfaces. - Graphical user interfaces.}

\terms{Design, Human Factors (Your general terms must be any of the
  following 16 designated terms: Algorithms, Management, Measurement,
  Documentation, Performance, Design, Economics, Reliability,
  Experimentation, Security, Human Factors, Standardization,
  Languages, Theory, Legal Aspects, Verification. See ~\cite{ACMTerms} for more details.)}

\keywords{Guides, instructions, formatting.}

\tolerance=400 
  % makes some lines with lots of white space, but 	
  % tends to prevent words from sticking out in the margin

\section{INTRODUCTION}

Spatial augmented reality was first created to project textures and illuminations to physical objects (cite raskar). Nowadays, it is mostly used for advertising, featuring projection on large building and commonly called ?projection mapping?. 
The unique ability of SAR is its penetration of the physical world. There are many uses, some are described in (cite mistry). It can be used to manipulate digital information (cite Wilson) or to manipulate the real world (cite onbody projection). Entertainment will benefit largely from SAR, the inclusion of digital games into the real world changes completely the aspect of it. (cite jeux SAR). 

In this demonstration, we propose to use SAR for physical artistic creation. We use computer graphics tools to enable easier and faster drawings. This demonstration is an evolution of the demonstration we did described in (papart) which was oriented for the general public. Here we focus on the tools and challenges of integrating digital information for drawing.  


\section{System... }


\section{Projection for drawing}
* 2D image projection
* Tablet + capture. 
* Papart -> Projection 3D

\section{Conclusion}



\end{document}
