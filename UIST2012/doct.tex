\documentclass{article}

\usepackage{times}
\usepackage{uist}

\begin{document}

% --- Copyright notice ---
\conferenceinfo{UIST'12}{October 7-10, 2012, Cambridge, MA, USA}
\CopyrightYear{2012}
\crdata{978-1-xxxx-xxxx-x}

% Uncomment the following line to hide the copyright notice
% \toappear{}
% ------------------------

\bibliographystyle{plain}

\title{Spatial augmented reality to enhance \\
       physical artistic creation.}

%%
%% Note on formatting authors at different institutions, as shown below:
%% Change width arg (currently 7cm) to parbox commands as needed to
%% accommodate widest lines, taking care not to overflow the 17.8cm line width.
%% Add or delete parboxes for additional authors at different institutions. 
%% If additional authors won't fit in one row, you can add a "\\"  at the
%% end of a parbox's closing "}" to have the next parbox start a new row.
%% Be sure NOT to put any blank lines between parbox commands!
%%

\author{
\parbox[t]{9cm}{\centering
	     {\em Jeremy Laviole}\\
	     Inria LaBRI\\
             ABC Corporation\\
	     1234 Anywhere Road\\
	     Anytown, NY 10027 USA\\
	     +1-212-555-1212\\
	     one@abc.com}
\parbox[t]{9cm}{\centering
	     {\em Martin Hachet}\\
         Inria LaBRI\\
	     5678 rue des Parapluies\\
	     99099 Cr\`{e}me de Menthe, FRANCE\\
	     +33-12-34-56-78\\
	     deux@uvw.xyz.fr}
}

\maketitle

\abstract
%Each paper should begin with an abstract, followed by a set of keywords, both
%placed in the left column of the first page under the left half of the title.
%All body text, such as this paragraph, should be set in 10 point Times Roman
%type, with 11 points between successive baselines.  (We will repeat that later
%in this document to make sure that you do not forget.)

\classification{H5.2 [Information interfaces and presentation]:
User Interfaces. - Graphical user interfaces.}

\terms{Design, Human Factors (Your general terms must be any of the
  following 16 designated terms: Algorithms, Management, Measurement,
  Documentation, Performance, Design, Economics, Reliability,
  Experimentation, Security, Human Factors, Standardization,
  Languages, Theory, Legal Aspects, Verification. See ~\cite{ACMTerms} for more details.)}

\keywords{Spatial augmented reality, user interfaces, fine arts}

\tolerance=400 
  % makes some lines with lots of white space, but 	
  % tends to prevent words from sticking out in the margin


%
%Id�e de plan.... 
%
%Introduction : SAR en g�n�ral, buts de la SAR, r�centes avanc�es. 
%
%Direction de la th�se pour en r�alit� augment�e. 
%
%Description du projet de th�se. Partie � d�tailler. 
%
%Travaux effectu�s ainsi que leur placement dans la th�se. Description de l'exp�rience au
%palais de la d�couverte. 
%
%Travaux en cours, possibilit�s et orientations. 
%Parler de tout ? 

%
%Objectifs 




\section{INTRODUCTION}

In the past 10 years, the field of augmented reality grows steadily. This is due to the availability 
of webcams in every laptop and mobile phone and the software such as ARToolKit.
However, the augmentations overlayed on reality is visualized though a screen. 
Even though it is convenient to navigate in the digital world by moving, for many 
applications we may want to have the hands free, or to interact directly with the 
real world and not by touching a screen.
In order to overcome these limitations, the digital information can be projected 
into reality. Consequently, the digital world has the possibility to merge with reality, 
it is spatial augmented reality (SAR). (cite raskar) 

Spatial augmented reality mixes the benefits of being directly linked to the reality. 
Consequently, the user experience can be enhanced and the digital part of the experience 
can be easily forgotten by the user. Conversely, all the SAR applications are bounded 
to real-life, it is harder to represent data behind real objects because the projection
support will importantly influence the visualization.
The projection support is generally assumed to be planar, or of a fixed shape. 
In (projo jeux) the freedom of movement of SAR is explored for video games applications. Another approach in (onbody proj), is to use SAR directly on the user's arm or a notepad to control a mobile phone or take notes.  
Recent work in SAR explore the projection support as the main interaction element. The 
augmented reality part will add new informations, some ideas of applications are described
in (sixth sense). The same idea is explored in (onbody projection), where the projection guides the user's movement. machin et al created a SAR application to ease the painting process, they propose tools to help the user to mix fresh paint and to apply the different layers in the right order. 

This kind of SAR in context is very promising, it is a first step for a more natural merge of the digital world and the real world. The idea is to create applications that fits into the real world, making it richer and more interactive. In this paper, I state the objectives which led to my past and current works. Then, I will conclude with a discussion of the future possibilities  

% Direction de la th�se pour en r�alit� augment�e. 

\section{Motivations for SAR for physical drawing.}

Coming from a background of computer science specialized in virtual reality and computer graphics, my first desire was to create tools and interactive applications in SAR. More specifically I wanted to use the power of real-time rendering for meaningful real-life applications. The solution I found is to use the power and tools of computer graphics to enhance physical drawings instead of digital drawings. 

There are many advantages to create digital drawings instead of physical ones. The most important one, is the possibility to go back and try again; which is obviously impossible in the physical world. I will not list all the possibilities offered by digital drawing, but some of them can be hard to reproduce for physical creation, such as the flexibility of digital layers, the copy/paste operations, filling regions with colors and textures and zooming operations.

Conversely, physical creation allows a direct contact between the artist and the creation. The result will be unique and will contain traces of all the errors corrected. The variety of tools comes from a legacy of thousands of years of artistic creations. Each pen or paint has its own smell and behaviour on the paper and tactile feeling. The bound between the artist, its tools and the resulting creation has history: from the acquisition of tools, to the whole creation process to achieve the desired result, or not. This bound may be different from digital creation, the choice of one digital brush, painting effect will not have any cost, and will generally not require to move from the computer. 

My main motivation is to create tools to make the creation and drawing process easier, faster, and enabling better results. Furthermore, I would like to push the possibilities of physical drawing, making enabling drawing creations that are hard to imagine such as ...
%anamorphic drawing or multi-perspective drawings.
In order to use the possibilities of real-time rendering, the artist will need to visualize and modify a 3D scene. In the next section, I present the system I created enabling projection on interactive paper sheets. 


\section{Spatial augmented reality and interactiveness}

\subsection*{The system}

For the creation of the system, we wanted to have as less constraints as possible. No constraint on the creation tools, no constraints on the creation support, and more importantly a large freedom of movement. We do no want to change the traditional setup, but seamlessly integrate the digital elements within it. 


The hardware system is composed of a small projector-camera (procam) set, and a depth camera (Kinect). We chose overhead projection to enable any support, and we use the Kinect to make the system interactive. The user interface is composed of tracked paper sheets, one or more is dedicated to the creation space. The paper or canvas is surrounded by markers (ARToolKitPlus) to achieve a high precision detection of its position. The whole drawing area is made tactile by the depth depth camera, even allowing 3D pointing interactions over it. Another paper sheet is used for the user interface, generally containing buttons and various indications.  


The aim of the system is to achieve a high speed tracking and more importantly high quality projection. Every projection default in the projection process, such as distortions or a bad estimation of the camera or projector parameters will be visible by the user and impacts the user experience. The hardware used is widely available for a moderate cost, the projector used is a DLP LED projector, the camera and depth camera are video game console accessories. 
 
\subsection*{3D projection and manipulation}

The firsts experiments involved the projection of photos inside the tracked paper sheet. The task was to create a drawing from the projected image, the projection acts as tracing paper. From this, we had some feedback: the tracking does not have to be fast during the drawing phase. When the user moves the drawing support, he or she can wait a second or two before drawing again. Another point is that any shift between the projection and the drawing will make the drawing much harder, and less comfortable. We also included tools to change the intensity of the projection, which was necessary to add any details to the drawings. 


The second application is the projection of a 3D scene. A 3D object is projected onto the paper sheet and behaves as if it was directly on the paper sheet. Consequently, if you see the front and want to see the back of the object, you just need to turn the paper sheet on the table. We also included rotate, scale and translate operations using the touch interface, allowing an easier modification of the scene. The point of view of scene is set to the user's head position, but generally we fixed it to make the system more robust and to keep it cheap. The 3D pointing was used to set a virtual light inside the virtual scene. The tangible interface and light placement enhance the perception of the 3D objects. We pushed it further by adding stereoscopic rendering. Using this system, we created an application and a drawing scenario for a general public exhibition in Paris. 

\subsection*{General public exhibition}

The exhibition lasted more than 3 month, and we were demonstrating and experimenting for more than 12 days. The aim of the demonstration was to explain our research to the general public and to make them experiment. We designed an drawing application using the projection of the 3D scene, and after  the description of the system, a visitor could come and create a drawing. In order to speed up the drawing phase, we used a simple toon shading for the rendering of the scene and a 3D model which was simple to draw. (photos de lapins !) The whole description of the system and elements of the presentation were using the touch and tangible interface; the mouse, keyboard and screen of the computer were hidden from the visitors.

 
The reception of the general public was good, and seemed magical for some visitors. When the paper sheets are on the table, the system is interactive, but without any paper sheet it is just a table with a few pencil and an eraser. We tested our drawing system nearly on 200 visitors, and nearly every of them were impressed by the quality of their own drawing. 


\section{Pushing the limits of the drawing}


The first hard drawing I explored with this is stereoscopic drawing. Since recently, stereoscopic screens are widely available, and is also getting widespread in the cinema. The creation of stereoscopic drawing is hard and repetitive. Two nearly identical drawings have to be made, and the visualization of these drawings require training for free view techniques or more generally a stereoscope. In (cite), we proposed a tool to create stereoscopic drawings from a pair of generated images. The images are created using the hardware and software described above. 

Just like any drawing, stereoscopic drawings require editions and adjustments. But unlike any drawing, it has to be done twice, and the differences between the two drawings will influence the depth perception of the result. We also included the possibility to take a photo of each drawing in order to have an anaglyph preview of the result. 

These experiments have raised many questions and observations. The first observation is on the quality of the drawing: the depth perception will be more influenced by the quality of the shading than by the left/right disparity. We are not sure that two drawings will be required, this question is complex. If just one drawing is done, and the second one is only the changed elements for stereoscopy, the objective to enhance the real world is not achieved any more: a digital reconstruction is required to visualize the result. 

\section{Perspectives of SAR for drawing}

\subsection*{Interaction techniques}

The inputs and interaction techniques described above allows direct and natural interactions. But, is is not precise for selection and manipulation of the digital information. In order to complement 
these we added a highly precise tablet input (wacom); it allows to overlay digital drawing over the physical one.                          

\subsection{New form of drawings}
% Definitions of drawing



% subsection applications... 


\section{Conclusion}

%%%	You can use bibtex if you like, but I've hardwired in these 
%%%	references to avoid sending you a separate .bib file.
\begin{thebibliography}{9}

%\bibitem{ACMTerms} How to Classify Works Using ACM�s Computing
%Classification System. {http://www.acm.org/class/how\_to\_use.html}.
%
%\bibitem{badenov91}  Badenov, B. Effects of prolonged use of WIMP user
%interfaces on Alces americana and Glaucomys volans.
%In {\em Proceedings of UIST '87}
%(February 30--April 1, Graceland, TN), ACM, NY, 1987, pp. 231--240.
%
%\bibitem{henry-etal92} Henry, T.R., Yeatts, A.K., Hudson, S.E., Myers, B.A.,
%and Feiner, S.K.  A nose gesture interface device: Extending virtual realities.
%{\em Presence 1}, 2 (Spring 1992), 258--261.
%
%\bibitem{GenderNeutral} Schwartz, M. Guidelines for Bias-Free Writing.
%Indiana University Press, Bloomington, IN, USA, 1995.
%
%\bibitem{zaranka81} Zaranka, W., Ed. {\em The Brand-X Anthology of Poetry:  A
%Parody Anthology.}  Apple-wood Books, Cambridge, MA, 1981.
\end{thebibliography}

\end{document}
